\documentclass{article}
\begin{document}

\noindent
Clint Valentine

\noindent
April 22 2015

\vfill

\noindent
First, we find the nullclines of the system:
$$\frac{dm_{i}}{dt}=-m_{i}+\frac{\alpha}{(1+p^{n}_{j})}+\alpha_{o}=0$$
$$\dot{p}=\dot{m}$$
$$\frac{dp_{i}}{dt}=-\beta(p_{i}-m_{i})=0$$
$$\dot{m}=\frac{\alpha}{(1 + \dot{p}^{n})}+\alpha_{o}$$
Next, we express the system in matrix form at the fixed point.
This is the Jacobian matrix of the system at the non-trivial fixed point:
$$
\mathbf{J}=
\left[{\begin{array}{*{20}c}
\frac{\partial{f_{1}}}{\partial{m}}&\frac{\partial{f_{1}}}{\partial{p}}\\
\frac{\partial{f_{2}}}{\partial{m}}&\frac{\partial{f_{2}}}{\partial{p}}
\end{array}}\right]=
\left[{\begin{array}{*{20}c}
-1&X\\
\beta&-\beta
\end{array}}\right]
$$
$$X\equiv-\frac{\alpha{}n\dot{p}^{n-1}}{(1+\dot{p}^n)^2}$$
$$\dot{p}=\frac{\alpha}{(1+\dot{p}^{n})}+\alpha_{o}$$
To derive the conditional for instability as expressed in the Elowitz and
Leibler (2000) paper we must find the the condition when the eigenvalues
have imaginary parts. The eigenvalues will have an imaginary part when
the discriminant of the Jacobian is negative.
$$\tau^{2}-4\Delta<0$$
$$(-1-\beta)^{2}-4(\beta-X\beta)<0$$
$$\frac{(\beta+1)^{2}}{\beta}<4(-X+1)$$
This expression is different than the Elowitz and Leibler (2000)
expression found in Box 1:
$$\frac{(\beta+1)^{2}}{\beta}<\frac{3X^{2}}{4+2X}$$

\vfill
\end{document}

